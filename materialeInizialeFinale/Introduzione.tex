% !TEX encoding = UTF-8
% !TEX TS-program = pdflatex
% !TEX root = ../Tesi.tex
% !TEX spellcheck = it-IT

%*******************************************************
% Introduzione
%*******************************************************
\cleardoublepage
\pdfbookmark{Introduzione}{introduzione}

\chapter*{Introduzione}

BINS (acronimo ricorsivo di "BINS Is Not Shopping") è un applicazione web liberamente disponibile, indicata soprattutto per commercializzare prodotti alimentari. Lo scopo di questo lavoro è di documentare l'applicazione usando l'UML (Unified Model Language).

La documentazione è articolata come segue.

\begin{description}
	
	\item[{\hyperref[cap:specifiche-progetto]{Il primo capitolo}}]
	contiene le specifiche del progetto descrivendo le caratteristiche e le categorie di utenti dell'applicazione.
	
	\item[{\hyperref[cap:diagramma]{Il secondo capitolo}}]
	spiega le operazioni, veramente semplici, per installare \LaTeX{} sul proprio calcolatore.
	
	\item[{\hyperref[cap:modello-relazionale]{Il terzo capitolo}}]
	spiega le operazioni, veramente semplici, per installare \LaTeX{} sul proprio calcolatore.
	
	\item[{\hyperref[cap:normalizzazione]{Il quarto capitolo}}]
	spiega le operazioni, veramente semplici, per installare \LaTeX{} sul proprio calcolatore.
	
	\item[{\hyperref[cap:codice-SQL]{Il quinto capitolo}}]
	spiega le operazioni, veramente semplici, per installare \LaTeX{} sul proprio calcolatore.
	
	\item[{\hyperref[cap:interfaccia-grafica]{Il sesto capitolo}}]
	spiega le operazioni, veramente semplici, per installare \LaTeX{} sul proprio calcolatore.
	
\end{description}

L'implementazione dell'applicazione è stata fatta utilizzando il framework \emph{Bootstrap} per ingegnerizzare al meglio il livello di presentazione.

Per l'application server si è usato \emph{Tomcat} e si è utilizzato il linguaggio di progammazione Java.
 Per interfacciare l'application server con il database si è utilizzato JDBC.
 
Il DBMS utilizzato è \emph{MySQL}.