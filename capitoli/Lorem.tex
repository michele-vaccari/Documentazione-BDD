% !TEX encoding = UTF-8
% !TEX TS-program = pdflatex
% !TEX root = ../Tesi.tex
% !TEX spellcheck = it-IT

%************************************************
\chapter{Lorem}
\label{cap:lorem}
%************************************************
% !TEX encoding = UTF-8
% !TEX TS-program = pdflatex
% !TEX root = ../Articolo.tex
% !TEX spellcheck = it-IT

%************************************************
\section{Lorem}
\label{sec:lorem}
%************************************************

\begin{figure}
\centering
\subfloat[Asia personas duo]
{\includegraphics[width=.45\columnwidth]{Lorem}} \quad
\subfloat[Pan ma signo]
{\label{fig:example-b}%
\includegraphics[width=.45\columnwidth]{Ipsum}} \\
\subfloat[Methodicamente o uno]
{\includegraphics[width=.45\columnwidth]{Dolor}} \quad
\subfloat[Titulo debitas]
{\includegraphics[width=.45\columnwidth]{Sit}}
\caption[Tu duo titulo debitas latente]{Tu duo titulo debitas latente}
\label{fig:example}
\end{figure}

Please note that the content of this section is just some dummy text. It isn't a real language.

Lorem ipsum dolor sit amet, consectetuer adipiscing elit. Ut purus elit, vestibulum ut, placerat ac, adipiscing vitae, felis. Curabitur dictum gravida mauris.

\subsection*{A subsection}

\lipsum[2]

\subsubsection*{A sub-subsection}

\lipsum[7]

\paragraph{A paragraph}
Lorem ipsum dolor sit amet, consectetuer adipiscing elit. Ut purus elit, vestibulum ut, placerat ac, adipiscing vitae, felis. Curabitur dictum gravida mauris. Nam arcu libero, nonummy eget, consectetuer id, vulputate a, magna.

\paragraph{Another paragraph}
Cras nec ante, pellentesque a nulla, cum sociis natoque penatibus et magnis dis parturient montes, nascetur ridiculus mus. Aliquam tincidunt urna

\bigskip

Donec aliquet, tortor sed accumsan bibendum, erat ligula aliquet magna, vitae ornare odio metus a mi. Morbi ac orci et nisl hendrerit mollis. Suspendisse ut massa. Cras nec ante. Pellentesque a nulla. Cum sociis natoque penatibus et magnis dis parturient montes, nascetur ridiculus mus. Aliquam tincidunt urna.

\begin{description}
\item[Mane] Lorem ipsum dolor sit amet, consectetuer adipiscing elit. 
\item[Tekel] Ut purus elit, vestibulum ut, placerat ac, adipiscing vitae, felis. Curabitur dictum gravida mauris.
\item[Fares] Nam arcu libero, nonummy eget, consectetuer 
id, vulputate a, magna.
\end{description}

Lorem ipsum dolor sit amet, consectetuer adipiscing elit. Ut purus elit, vestibulum ut, placerat ac, adipiscing vitae, felis. Curabitur dictum gravida mauris. Nam arcu libero, nonummy eget, consectetuer id, vulputate a, magna.


\subsection*{Some formulas}

Una formula in linea viene incorporata nel testo: $\lim_{n \to \infty}\sum_{k=1}^n \frac{1}{k^2} = \frac{\pi^2}{6}$, per esempio. Come si osserva, \LaTeX{} fa \emph{il possibile} per comprimerla e modificare il meno possibile l'interlinea nel capoverso che la contiene.
Una formula in display viene invece composta da \LaTeX{} su linee a parte, separate dal contesto con adeguati spazi bianchi per metterla in mostra e farla risaltare sulla pagina.
\begin{equation}
\lim_{n \to \infty}\sum_{k=1}^n \frac{1}{k^2}= \frac{\pi^2}{6}
\end{equation}
Come si osserva, ora la formula risulta centrata, non compressa, e tutti i suoi elementi occupano il giusto spazio con un risultato finale di grande respiro.

\begin{table}
\caption{Lorem ipsum dolor sit amet}
\centering
\begin{tabular}{ll}
\toprule
\textbf{Alkaloid} & \textbf{Origin} \\
\midrule
atropine & belladonna \\
morphine & poppy \\
nicotine & tobacco \\
\bottomrule
\end{tabular}
\end{table}



Integer tempus convallis augue. Etiam facilisis. Nunc elementum fermentum wisi. Aenean placerat. Ut imperdiet, enim sed gravida sollicitudin, felis odio placerat quam, ac pulvinar elit purus eget enim. 

\begin{equation}
\int_a^{a+T}f(x)\,dx= \int_0^T f(x)\,dx 
\qquad
\oint f(z)\,dz=2\pi i
\end{equation}

Nulla malesuada porttitor diam. Donec felis erat, congue non, volutpat at, tincidunt tristique, libero. Vivamus viverra fermentum felis. Donec non- ummy pellentesque ante.

\begin{equation}
f(x_1,\dots,x_n)=  \prod_{k=1}^n x_k 
\qquad
\sum_{k=1}^n x_k^2=1
\qquad
\biggl(\sum_n x_n^2\biggr)^{1/2} 
\end{equation}

\lipsum[2]

\begin{equation}
\begin{bmatrix} 
a_{11} & \dots & a_{1n} \\ 
a_{21} & \dots & a_{2n} \\ 
\hdotsfor{3} \\ 
a_{n1} & \dots & a_{nn} 
\end{bmatrix}
\end{equation}

\lipsum[4]

\begin{equation}
\lim_{x\to 0}
\frac{\sin x}{x}=1 \qquad
\lim_{n\to +\infty}f_n=\delta
\end{equation}

\section{Ipsum}

Fusce mauris. Vestibulum luctus nibh at lectus. Sed bibendum, nulla a faucibus semper, leo velit ultricies tellus, ac venenatis arcu wisi vel nisl.

\begin{equation}
n!=
\begin{cases} 
1       & \text{if $n=0$} \\ 
n(n-1)! & \text{if $n\ge 1$} 
\end{cases} 
\end{equation}

Ut lectus eros, malesuada sit amet, fermentum eu, sodales cursus, magna. Quisque vehicula, urna sed auctor, pede lorem egestas dui, et convallis elit erat sed nulla.

\begin{equation} 
x_G=
\frac{\displaystyle
      \sum_{i=1}^n m_ix_i}
{\displaystyle\sum_{i=1}^n m_i}
\end{equation}

\section{Dolor}

\lipsum[6]

\begin{equation}
\kappa =\frac{\xi}{E_{\textrm{max}}}
\qquad
E_{\textup{max}} =\frac{2 m_{\textup{e}} \beta^2\gamma^2 }{1 +2\gamma m_{\textup{e}}/m_{\textrm{x}} + ( m_{\textup{e}}/m_{\textup{x}})^2}
\end{equation}

\lipsum[8]