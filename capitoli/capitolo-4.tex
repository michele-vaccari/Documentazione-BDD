% !TEX encoding = UTF-8
% !TEX TS-program = pdflatex
% !TEX root = ../Tesi.tex
% !TEX spellcheck = it-IT

%************************************************
\chapter{Normalizzazione}
\label{cap:normalizzazione}
%************************************************

\section{Cenni teorici}
Per normalizzazione si intende il procedimento che ha lo scopo di eliminare la ridondanza e le eventuali incoerenze del database.
Il processo di normalizzazione sottopone uno schema relazionale a una serie di test per certificare se soddisfa una data forma normale.

	\subsection{Prima forma normale (1NF)}
	Si dice che uno schema relazionale è in prima forma normale se il dominio di un attributo comprende valori atomici e se il valore di un qualsiasi attributo in una tupla sia un singolo valore del dominio, ovvero:
	\begin{itemize}
		
		\item
		Tutte le righe della tabella contengono lo stesso numero di colonne;
		
		\item
		Gli attributi rappresentano informazioni elementari;
		
		\item
		I valori che compaiono in una colonna sono dello stesso tipo, cioè appartengono allo stesso dominio;
		
		\item
		Ogni riga è diversa da tutte le altre, cioè non ce ne possono essere due con gli stessi valori nelle colonne;
		
		\item
		L'ordine con il quale le righe compaiono nella tabella è irrilevante.
		
	\end{itemize}
	La prima forma normale ($1NF$) è già parte integrante della definizione formale di relazione nel modello relazionale.
	
	\subsection{Seconda forma normale (2NF)}
	Si dice che uno schema relazionale è in seconda forma normale ($2NF$) quando è in prima forma normale ($1NF$) e tutti i suoi attributi non-chiave dipendono dall'intera chiave, cioè non possiede attributi che dipendono soltanto da una parte della chiave.
	
	La seconda forma normale elimina la dipendenza parziale degli attributi dalla chiave e riguarda il caso di relazioni con chiavi composte, cioè formate da più attributi.
	
	\subsection{Terza forma normale (3NF)}
	Si dice che uno schema relazionale è in terza forma normale ($3NF$) quando è in seconda forma normale ($2NF$) e tutti i suoi attributi non-chiave dipendono direttamente dalla chiave, cioè non possiede attributi non-chiave che dipendono da altri attributi non-chiave.
	
	La terza forma normale elimina la dipendenza transitiva degli attributi dalla chiave.
	
\section{Processo di normalizzazione}
Partendo dalla figura~\vref{f:trad-assoc}, si analizzano le varie chiavi e le varie dipendenze funzionali per ridurre lo schema relazionale precedentemente descritto in $2NF$ e in $3NF$.\\
\\
Si analizza la relazione FATTURA e si nota che essa risulta essere già in seconda forma normale ($2NF$);
infatti essendoci un solo attributo chiave (\emph{Id}) tutti gli altri attributi dipendono funzionalmente da quest'ultimo.
Infine non essendoci dipendenze funzionali transitive, tale relazione risulta essere anche in terza forma normale ($3NF$).

Lo stesso ragionamento si può applicare alle relazioni SCADENZA, INDIRIZZO\_DI\_SPEDIZIONE e TRACKING. \\
\\
Si considerano ora le seguenti relazioni:
\begin{itemize}
	
	\item
	MARCHIO
	
	\item
	CARATTERISTICA
	
	\item
	REPARTO
	
	\item
	UTENTE\_REGISTRATO
	
	\item
	PAGAMENTO
	
\end{itemize}
Tutte le relazioni sono formate da un unico attributo chiave e da una chiave esterna, pertanto risultano essere in $3NF$.\\
\\
La relazione UTENTE\_AMMINISTRATORE è formata da un'unico attributo chiave e presenta l'attributo \emph{Email\_USA} non-primo che dipende funzionalmente, direttamente e completamente dalla chiave;
quindi ne segue che la relazione è in $3NF$.\\
\\
Si considerano ora le seguenti relazioni:
\begin{itemize}
	
	\item
	PRODOTTO
	
	\item
	COUPON
	
\end{itemize}
Entrambe le relazioni sono formate da un unico attributo chiave e da due chiavi esterne, pertanto risultano essere in $3NF$. \\
\\
La relazione ORDINE presenta un attributo chiave e tre chiavi esterne, quindi risulta essere in $3NF$. \\
\\
La relazione RICETTA presenta due attributi chiave e una esterna, quindi risulta essere in $3NF$. \\
\\
La relazioni LISTA\_DELLA\_SPESA e INDIRIZZO presentano due attributi chiave, quindi risulta essere in $3NF$. \\
\\
Per quanto riguarda le relazioni:
\begin{itemize}
	
	\item
	COMPONE
	
	\item
	CONTENUTO\_IN\_CARRELLO
	
	\item
	HA\_PARTICOLARE
	
	\item
	GESTISCE\_PRODOTTI
	
\end{itemize}
Tali relazioni sono formate da due attributi chiave che sono chiavi esterne, pertanto risultano essere in $3NF$. \\
\\
Le relazioni CONTENUTO\_IN\_LISTA e IMPIEGATO\_IN presentano tre attributi chiave che sono chiavi esterne, quindi risultano essere in $3NF$.\\
\\
Dall'analisi svolta si deduce che la figura~\vref{f:trad-assoc-m-n} è in terza forma normale ($3NF$).
