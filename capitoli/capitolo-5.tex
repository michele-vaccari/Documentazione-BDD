% !TEX encoding = UTF-8
% !TEX TS-program = pdflatex
% !TEX root = ../Tesi.tex
% !TeX spellcheck = it_IT

%************************************************
\chapter{Codice SQL}
\label{cap:codice-sql}
%************************************************

\section{DDL}
Si riporta il codice SQL utilizzato per creare le tabelle che costituiscono la base di dati e i vincoli di integrità referenziale.

\subsection{Utente}
Struttura della tabella \emph{UTENTE}:

\begin{lstlisting}
CREATE TABLE IF NOT EXISTS `UTENTE` (
    `ID_Utente` INT UNSIGNED NOT NULL,
	`Email` VARCHAR(30) NOT NULL,
	`Password` VARCHAR(16) NOT NULL,
	`Nome` VARCHAR(30) NOT NULL,
	`Cognome` VARCHAR(30) NOT NULL,
	`Tipo` VARCHAR(1) NOT NULL,
	`Attivo` VARCHAR(1) NOT NULL DEFAULT 'Y',
	PRIMARY KEY (`ID_Utente`)
) ENGINE = InnoDB;
\end{lstlisting}

\subsection{Amministratore}
Struttura della tabella \emph{AMMINISTRATORE}:

\begin{lstlisting}
CREATE TABLE IF NOT EXISTS `AMMINISTRATORE` (
	`ID_Utente` INT UNSIGNED NOT NULL,
	PRIMARY KEY (`ID_Utente`),
	FOREIGN KEY (`ID_Utente`) REFERENCES UTENTE(`ID_Utente`)
) ENGINE = InnoDB;
\end{lstlisting}

\newpage

\subsection{Registrato}
Struttura della tabella \emph{REGISTRATO}:

\begin{lstlisting}
CREATE TABLE IF NOT EXISTS `REGISTRATO` (
	`ID_Utente` INT UNSIGNED NOT NULL,
	`ID_Amministratore` INT UNSIGNED NOT NULL,
	`Telefono` VARCHAR(15) NOT NULL,
	`Indirizzo` VARCHAR(50) NOT NULL,
	PRIMARY KEY (`ID_Utente`),
	FOREIGN KEY (`ID_Utente`) REFERENCES UTENTE(`ID_Utente`),
	FOREIGN KEY (`ID_Amministratore`) REFERENCES AMMINISTRATORE(`ID_Utente`)
) ENGINE = InnoDB;
\end{lstlisting}

\subsection{Arbitro}
Struttura della tabella \emph{ARBITRO}:

\begin{lstlisting}
CREATE TABLE IF NOT EXISTS `ARBITRO` (
	`ID_Registrato` INT UNSIGNED NOT NULL,
	`Data_di_nascita` VARCHAR(10) NOT NULL,
	`Nazionalita` VARCHAR(20) NOT NULL,
	`Foto` VARCHAR(50) NOT NULL,
	`Carriera` TEXT(5000),
	PRIMARY KEY (`ID_Registrato`),
	FOREIGN KEY (`ID_Registrato`) REFERENCES REGISTRATO(`ID_Utente`)
) ENGINE = InnoDB;
\end{lstlisting}

\subsection{Gestore Squadra}
Struttura della tabella \emph{GESTORE\_SQUADRA}:

\begin{lstlisting}
CREATE TABLE IF NOT EXISTS `GESTORE_SQUADRA` (
	`ID_Registrato` INT UNSIGNED NOT NULL,
	PRIMARY KEY (`ID_Registrato`),
	FOREIGN KEY (`ID_Registrato`) REFERENCES REGISTRATO(`ID_Utente`)
) ENGINE = InnoDB;
\end{lstlisting}

\newpage

\subsection{Torneo}
Struttura della tabella \emph{TORNEO}:

\begin{lstlisting}
CREATE TABLE IF NOT EXISTS `TORNEO` (
	`ID_Torneo` INT UNSIGNED NOT NULL,
	`ID_Amministratore` INT UNSIGNED NOT NULL,
	`Tipologia` CHAR(1) NOT NULL,
	`Nome` VARCHAR(30) NOT NULL,
	`Descrizione` TEXT(5000),
	PRIMARY KEY (`ID_Torneo`),
	FOREIGN KEY (`ID_Amministratore`) REFERENCES AMMINISTRATORE(`ID_Utente`)
) ENGINE = InnoDB;
\end{lstlisting}

\subsection{Partita}
Struttura della tabella \emph{PARTITA}:

\begin{lstlisting}
CREATE TABLE IF NOT EXISTS `PARTITA` (
	`ID_Partita` INT UNSIGNED NOT NULL,
	`Tipo` CHAR(1) NOT NULL,
	`Luogo` VARCHAR(30) NOT NULL,
	`Data_partita` VARCHAR(30) NOT NULL,
	`Ora` VARCHAR(5) NOT NULL,
	PRIMARY KEY (`ID_Partita`)
) ENGINE = InnoDB;
\end{lstlisting}

\subsection{Referto}
Struttura della tabella \emph{REFERTO}:

\begin{lstlisting}
CREATE TABLE IF NOT EXISTS `REFERTO` (
	`ID_Referto` INT UNSIGNED NOT NULL,
	`ID_Partita` INT UNSIGNED NOT NULL,
	`ID_Arbitro` INT UNSIGNED NOT NULL,
	`Ora_inizio` VARCHAR(5),
	`Ora_fine` VARCHAR(5),
	`Risultato` VARCHAR(10),
	`Compilato` VARCHAR(1) NOT NULL,
	PRIMARY KEY (`ID_Referto`),
	FOREIGN KEY (`ID_Partita`) REFERENCES PARTITA(`ID_Partita`),
	FOREIGN KEY (`ID_Arbitro`) REFERENCES ARBITRO(`ID_Registrato`)
) ENGINE = InnoDB;
\end{lstlisting}

\newpage

\subsection{Squadra}
Struttura della tabella \emph{SQUADRA}:

\begin{lstlisting}
CREATE TABLE IF NOT EXISTS `SQUADRA` (
	`ID_Squadra` INT UNSIGNED NOT NULL,
	`ID_Gestore` INT UNSIGNED NOT NULL,
	`Compilata` VARCHAR(1) NOT NULL,
	`Nome_squadra` VARCHAR(30) NOT NULL,
	`Descrizione` TEXT(5000),
	`Sede` VARCHAR(30),
	`Logo_squadra` VARCHAR(50),
	`Immagine_squadra` VARCHAR(50),
	`Nome_sponsor` VARCHAR(30),
	`Logo_sponsor` VARCHAR(50),
	PRIMARY KEY (`ID_Squadra`),
	FOREIGN KEY (`ID_Gestore`)
	            REFERENCES GESTORE_SQUADRA(`ID_Registrato`)
) ENGINE = InnoDB;
\end{lstlisting}

\subsection{Classifica}
Struttura della tabella \emph{CLASSIFICA}:

\begin{lstlisting}
CREATE TABLE IF NOT EXISTS `CLASSIFICA` (
	`ID_Torneo` INT UNSIGNED NOT NULL,
	`ID_Squadra` INT UNSIGNED NOT NULL,
	`Punti` INT UNSIGNED DEFAULT 0,
	`Vittorie` INT UNSIGNED DEFAULT 0,
	`Sconfitte` INT UNSIGNED DEFAULT 0,
	`Pareggi` INT UNSIGNED DEFAULT 0,
	`Partite` INT UNSIGNED DEFAULT 0,
	`Goal_fatti` INT UNSIGNED DEFAULT 0,
	`Goal_subiti` INT UNSIGNED DEFAULT 0,
	PRIMARY KEY (`ID_Torneo`, `ID_Squadra`),
	FOREIGN KEY (`ID_Torneo`) REFERENCES TORNEO(`ID_Torneo`),
	FOREIGN KEY (`ID_Squadra`) REFERENCES SQUADRA(`ID_Squadra`)
) ENGINE = InnoDB;
\end{lstlisting}

\newpage

\subsection{Giocatore}
Struttura della tabella \emph{GIOCATORE}:

\begin{lstlisting}
CREATE TABLE IF NOT EXISTS `GIOCATORE` (
	`ID_Giocatore` INT UNSIGNED NOT NULL,
	`ID_Squadra` INT UNSIGNED NOT NULL,
	`Attivo` VARCHAR(1) DEFAULT 'Y',
	`Numero_maglia` INT UNSIGNED,
	`Ruolo` VARCHAR(50) NOT NULL,
	`Nome` VARCHAR(30) NOT NULL,
	`Cognome` VARCHAR(30) NOT NULL,
	`Data_nascita` VARCHAR(10) NOT NULL,
	`Nazionalita` VARCHAR(20) NOT NULL,
	`Foto` VARCHAR(50) NOT NULL,
	`Descrizione` TEXT(5000),
	`Goal` INT UNSIGNED DEFAULT 0,
	`Ammonizioni` INT UNSIGNED DEFAULT 0,
	`Espulsioni` INT UNSIGNED DEFAULT 0,
	PRIMARY KEY (`ID_Giocatore`),
	FOREIGN KEY (`ID_Squadra`) REFERENCES SQUADRA(`ID_Squadra`)
) ENGINE = InnoDB;
\end{lstlisting}

\subsection{Marcatore}
Struttura della tabella \emph{MARCATORE}:

\begin{lstlisting}
CREATE TABLE IF NOT EXISTS `MARCATORE` (
	`ID_Referto` INT UNSIGNED NOT NULL,
	`ID_Giocatore` INT UNSIGNED NOT NULL,
	`Goal` INT UNSIGNED DEFAULT 0,
	PRIMARY KEY (`ID_Referto`, `ID_Giocatore`),
	FOREIGN KEY (`ID_Referto`) REFERENCES REFERTO(`ID_Referto`),
	FOREIGN KEY (`ID_Giocatore`)
	            REFERENCES GIOCATORE(`ID_Giocatore`)
) ENGINE = InnoDB;
\end{lstlisting}

\subsection{Partecipano}
Struttura della tabella \emph{PARTECIPANO}:

\begin{lstlisting}
CREATE TABLE IF NOT EXISTS `PARTECIPANO` (
	`ID_Squadra` INT UNSIGNED NOT NULL,
	`ID_Torneo` INT UNSIGNED NOT NULL,
	PRIMARY KEY (`ID_Squadra`, `ID_Torneo`),
	FOREIGN KEY (`ID_Squadra`) REFERENCES SQUADRA(`ID_Squadra`),
	FOREIGN KEY (`ID_Torneo`) REFERENCES TORNEO(`ID_Torneo`)
) ENGINE = InnoDB;
\end{lstlisting}

\subsection{Cartellino}
Struttura della tabella \emph{CARTELLINO}:

\begin{lstlisting}
CREATE TABLE IF NOT EXISTS `CARTELLINO` (
	`ID_Referto` INT UNSIGNED NOT NULL,
	`ID_Giocatore` INT UNSIGNED NOT NULL,
	`Numero` INT UNSIGNED DEFAULT 0,
	`Ammonizione` VARCHAR(1),
	`Espulsione` VARCHAR(1),
	PRIMARY KEY (`ID_Referto`, `ID_Giocatore`),
	FOREIGN KEY (`ID_Referto`) REFERENCES REFERTO(`ID_Referto`),
	FOREIGN KEY (`ID_Giocatore`)
	            REFERENCES GIOCATORE(`ID_Giocatore`)
) ENGINE = InnoDB;
\end{lstlisting}

\subsection{Scelto}
Struttura della tabella \emph{SCELTO}:

\begin{lstlisting}
CREATE TABLE IF NOT EXISTS `SCELTO` (
	`ID_Partita` INT UNSIGNED NOT NULL,
	`ID_Arbitro` INT UNSIGNED NOT NULL,
	`ID_Amministratore` INT UNSIGNED NOT NULL,
	PRIMARY KEY (`ID_Partita`, `ID_Arbitro`, `ID_Amministratore`),
	FOREIGN KEY (`ID_Partita`) REFERENCES PARTITA(`ID_Partita`),
	FOREIGN KEY (`ID_Arbitro`) REFERENCES ARBITRO(`ID_Registrato`),
	FOREIGN KEY (`ID_Amministratore`)
	            REFERENCES AMMINISTRATORE(`ID_Utente`)
) ENGINE = InnoDB;
\end{lstlisting}

\subsection{Formazione}
Struttura della tabella \emph{FORMAZIONE}:

\begin{lstlisting}
CREATE TABLE IF NOT EXISTS `FORMAZIONE` (
	`ID_Referto` INT UNSIGNED NOT NULL,
	`ID_Squadra` INT UNSIGNED NOT NULL,
	`ID_Giocatore` INT UNSIGNED NOT NULL,
	`Riserva` VARCHAR(1) NOT NULL,
	PRIMARY KEY (`ID_Referto`, `ID_Squadra`, `ID_Giocatore`),
	FOREIGN KEY (`ID_Referto`) REFERENCES REFERTO(`ID_Referto`),
	FOREIGN KEY (`ID_Squadra`) REFERENCES SQUADRA(`ID_Squadra`),
	FOREIGN KEY (`ID_Giocatore`)
	            REFERENCES GIOCATORE(`ID_Giocatore`)
) ENGINE = InnoDB;
\end{lstlisting}

\newpage

\subsection{Giocano}
Struttura della tabella \emph{GIOCANO}:

\begin{lstlisting}
CREATE TABLE IF NOT EXISTS `GIOCANO` (
	`ID_Partita` INT UNSIGNED NOT NULL,
	`ID_SquadraA` INT UNSIGNED NOT NULL,
	`ID_SquadraB` INT UNSIGNED NOT NULL,
	`ID_Torneo` INT UNSIGNED NOT NULL,
	`Nome_partita` VARCHAR(30) NOT NULL,
	PRIMARY KEY (`ID_Partita`, `ID_SquadraA`, `ID_SquadraB`,
	             `ID_Torneo`),
	FOREIGN KEY (`ID_Partita`) REFERENCES PARTITA(`ID_Partita`),
	FOREIGN KEY (`ID_SquadraA`) REFERENCES SQUADRA(`ID_Squadra`),
	FOREIGN KEY (`ID_SquadraB`) REFERENCES SQUADRA(`ID_Squadra`),
	FOREIGN KEY (`ID_Torneo`) REFERENCES TORNEO(`ID_Torneo`)
) ENGINE = InnoDB;
\end{lstlisting}

\section{Interrogazioni}
Tra le molte interrogazioni utilizzate dall'applicazione web riguardo il database precedente si riportano quelle che meritano un commento.

\subsection{INSERT}

\subsection*{Inserimento di un nuovo arbitro}
Quando l'amministratore inserisce un nuovo arbitro, viene settato nella tabella \emph{UTENTE} l'attributo \emph{Tipo} ``R'' (dall'inglese Refree). Le query che vengono eseguite sono tre perché \emph{ARBITRO} è una specializzazione dell'entità \emph{REGISTRATO} che a sua volta è una specializzazione dell'entità \emph{UTENTE}.

\subsubsection*{Inserimento utente}

\begin{lstlisting}
INSERT INTO utente (ID_Utente, Nome, Cognome, Password, Email, Tipo,
                    Attivo)
VALUES (id,
        'Conversion.getDatabaseString(nome)',
        'Conversion.getDatabaseString(cognome)',
        'Conversion.getDatabaseString(password)',
        'Conversion.getDatabaseString(email)',
        'R',
        'Y');
\end{lstlisting}

\newpage

\subsubsection*{Inserimento registrato}

\begin{lstlisting}
INSERT INTO registrato (ID_Utente, Telefono, Indirizzo,
                        ID_Amministratore)
VALUES (id,
        'Conversion.getDatabaseString(telefono)',
        'Conversion.getDatabaseString(indirizzo)',
        'Conversion.getDatabaseString(Admin)');
\end{lstlisting}

\subsubsection*{Aggiornamento di un arbitro}

\begin{lstlisting}
INSERT INTO arbitro (ID_Registrato, Foto, Data_nascita, Nazionalita,
                     Carriera)
VALUES (id,
        'Conversion.getDatabaseString(foto)',
        'Conversion.getDatabaseString(dataNascita)',
        'Conversion.getDatabaseString(nazionalita)',
        'Conversion.getDatabaseString(carriera)');
\end{lstlisting}

\subsection*{Inserimento dei marcatori di una partita}
Nella creazione del referto viene utilizzata la query seguente per inserire nella tabella \emph{MARCATORE} tutti i giocatori che hanno segnato dei goal nella partita a cui fa riferimento il referto.

\begin{lstlisting}
INSERT INTO marcatore (ID_Referto, ID_Giocatore, Goal)
VALUES (IDReferto, IDGiocatore, goal);
\end{lstlisting}

\subsection{UPDATE}

\subsection*{Primo accesso all'applicazione del gestore di una squadra}
Quando il gestore di una squadra, dopo essere stato inserito nel sistema dall'amministratore, effettua il primo accesso all'applicazione deve compilare i dati della sua squadra. L'aggiornamento con i nuovi dati inseriti avviene eseguendo il codice seguente:

\begin{lstlisting}
UPDATE squadra
SET Nome_squadra = 'Conversion.getDatabaseString(nomeSquadra)',
    Logo_squadra = 'Conversion.getDatabaseString(logoSquadra)',    
    Immagine_squadra = 'Conversion.getDatabaseString(immagineSquadra)',
    Nome_sponsor = 'Conversion.getDatabaseString(nomeSponsor)',
    Logo_sponsor = 'Conversion.getDatabaseString(logoSponsor)',
    Sede = 'Conversion.getDatabaseString(sede)',
    Descrizione = 'Conversion.getDatabaseString(descrizione)',
    Compilata = 'Y'
WHERE ID_Gestore = gestore;
\end{lstlisting}

\subsection*{Eliminazione di un utente dall'applicazione}
L'amministratore ha la possibilità di rimuovere i privilegi di accesso all'applicazione ad un utente registrato da lui creato in precedenza. La cancellazione che viene effettuata è una cancellazione logica in quanto viene cambiata la visibilità dell'utente attraverso l'attributo \emph{Attivo}.
La query che effettua la cancellazione logica è la seguente:

\begin{lstlisting}
UPDATE utente
SET Attivo = 'N'
WHERE Email = 'Conversion.getDatabaseString(email)';
\end{lstlisting}

\newpage

\subsection{SELECT}

\subsection*{Ricerca formazioni non ancora compilate per un gestore}
Quando l'amministratore del sistema assegna un arbitro ad una partita di un torneo, il gestore della squadra deve compilare la formazione per la partita. La query seguente viene eseguita quando il gestore della pagina visualizza le partite per cui deve ancora compilare una formazione:

\begin{lstlisting}
SELECT P.Data_partita, R.ID_Referto, G.*, R.Compilato,
       T.Nome AS nomeTorneo
FROM referto AS R, giocano AS G, torneo AS T, partita AS P
WHERE R.Compilato='N' AND G.ID_Partita = R.ID_Partita AND
      G.ID_Torneo = T.ID_Torneo AND R.ID_Partita = P.ID_Partita AND
      G.ID_Partita = P.ID_Partita AND 
      (G.ID_SquadraA IN (SELECT ID_SQUADRA
                         FROM squadra
                         WHERE ID_Gestore = gestore AND
                         ID_Squadra <> 0) OR
       G.ID_SquadraB IN (SELECT ID_SQUADRA FROM squadra
                         WHERE ID_Gestore = gestore AND
                         ID_Squadra <> 0)) AND
      R.ID_Referto NOT IN (SELECT ID_Referto
                           FROM formazione
                           WHERE ID_Squadra = (SELECT ID_Squadra
                                               FROM squadra
                                               WHERE ID_Gestore =
                                               gestore ) )
ORDER BY ID_Referto;
\end{lstlisting}

\subsection*{Visualizzazione dei referti assegnati ad un arbitro da un amministratore}
Quando l'amministratore del sistema deve sorteggiare un arbitro per una determinata partita riguardante una certa fase di un determinato torneo, deve poter visualizzare soltanto gli incontri per cui il sorteggio non sia già stato effettuato. La query seguente visualizza la lista di tutte le partite assegnate ad un arbitro dall'amministratore:

\begin{lstlisting}
SELECT REF.ID_Referto, REF.Risultato, REF.Ora_inizio, REF.Ora_fine,
       REF.ID_Partita, REF.Risultato, REF.Compilato,
       U.Nome AS nomeArbitro, U.Cognome AS CognomeArbitro,
       T.ID_Torneo, T.Tipologia as TipoTorneo,
       T.Nome as NomeTorneo, P.Data_Partita, P.Luogo
FROM referto AS REF, utente AS U, registrato AS REG, giocano AS G,
     torneo AS T, partita AS P
WHERE U.Attivo = 'Y' AND U.Tipo = 'R' AND
      U.ID_Utente = REG.ID_Utente AND U.ID_Utente = REF.ID_Arbitro AND
      G.ID_partita = REF.ID_Partita AND T.ID_Torneo = G.ID_Torneo AND
      P.ID_Partita = G.ID_Partita AND REG.ID_Amministratore = admin
ORDER BY REF.ID_Referto
\end{lstlisting}

\subsection*{Visualizzazione dei cartellini dei giocatori in un referto di una partita}
Quando l'utente pubblico va a visionare il referto relativo di una determinata partita, vengono estratte dalla tabella \emph{CARTELLINO} le ammonizioni e le espulsioni di tutti i giocatori per la partita selezionata. La query seguente visualizza i cartellini per ogni giocatore presente in un referto:

\begin{lstlisting}
SELECT G.Nome, G.Cognome, G.Foto, G.ID_Giocatore, G.ID_Squadra,
       C.Ammonizione AS FlagAmmonito, C.Espulsione AS FlagEspulso
FROM giocatore AS G, cartellino AS C
WHERE C.ID_Giocatore = G.ID_Giocatore AND
      C.ID_Referto = IDReferto AND (G.ID_Squadra = IDSquadraA OR
                                    G.ID_Squadra = IDSquadraB);
\end{lstlisting}