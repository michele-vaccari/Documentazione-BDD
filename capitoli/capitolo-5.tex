% !TEX encoding = UTF-8
% !TEX TS-program = pdflatex
% !TEX root = ../Tesi.tex
% !TeX spellcheck = it_IT

%************************************************
\chapter{Codice SQL}
\label{cap:codice-sql}
%************************************************

\section{DDL}
Si riporta il codice SQL utilizzato per creare le tabelle che costituiscono la base di dati e i vincoli di integrità referenziale.

\subsection{Utente}
Struttura della tabella \emph{UTENTE}:

\begin{lstlisting}
CREATE TABLE IF NOT EXISTS `UTENTE` (
	`Email` VARCHAR(30) NOT NULL,
	`Password` VARCHAR(32) NOT NULL,
	`Chiave_privata` VARCHAR(32) NOT NULL,
	`Nome` VARCHAR(30) NOT NULL,
	`Cognome` VARCHAR(30) NOT NULL,
	`Tipo` VARCHAR(1) NOT NULL,
	`Attivo` BOOLEAN NOT NULL DEFAULT TRUE,
	PRIMARY KEY (`Email`)
) ENGINE = InnoDB;
\end{lstlisting}

\subsection{Amministratore}
Struttura della tabella \emph{AMMINISTRATORE}:

\begin{lstlisting}
CREATE TABLE IF NOT EXISTS `AMMINISTRATORE` (
	`Email_U` VARCHAR(30) NOT NULL,
	PRIMARY KEY (`Email_U`),
	FOREIGN KEY (`Email_U`) REFERENCES UTENTE(`Email`)
) ENGINE = InnoDB;
\end{lstlisting}

\newpage

\subsection{Registrato}
Struttura della tabella \emph{REGISTRATO}:

\begin{lstlisting}
CREATE TABLE IF NOT EXISTS `REGISTRATO` (
	`Email_U` VARCHAR(30) NOT NULL,
	`Email_A` VARCHAR(30) NOT NULL,
	`Telefono` VARCHAR(15) NOT NULL,
	`Indirizzo` VARCHAR(50) NOT NULL,
	PRIMARY KEY (`Email_U`),
	FOREIGN KEY (`Email_U`) REFERENCES UTENTE(`Email`),
	FOREIGN KEY (`Email_A`) REFERENCES AMMINISTRATORE(`Email_U`)
) ENGINE = InnoDB;
\end{lstlisting}

\subsection{Arbitro}
Struttura della tabella \emph{ARBITRO}:

\begin{lstlisting}
CREATE TABLE IF NOT EXISTS `ARBITRO` (
	`Email_R` VARCHAR(30) NOT NULL,
	`Data_di_nascita` DATE NOT NULL,
	`Nazionalita` VARCHAR(50) NOT NULL,
	`Foto` VARCHAR(50) NOT NULL,
	PRIMARY KEY (`Email_R`),
	FOREIGN KEY (`Email_R`) REFERENCES REGISTRATO(`Email_U`)
) ENGINE = InnoDB;
\end{lstlisting}

\subsection{Gestore Squadra}
Struttura della tabella \emph{GESTORE\_SQUADRA}:

\begin{lstlisting}
CREATE TABLE IF NOT EXISTS `GESTORE_SQUADRA` (
	`Email_R` VARCHAR(30) NOT NULL,
	PRIMARY KEY (`Email_R`),
	FOREIGN KEY (`Email_R`) REFERENCES REGISTRATO(`Email_U`)
) ENGINE = InnoDB;
\end{lstlisting}

\subsection{Torneo}
Struttura della tabella \emph{TORNEO}:

\begin{lstlisting}
CREATE TABLE IF NOT EXISTS `TORNEO` (
	`Id` INT UNSIGNED NOT NULL,
	`Email_A` VARCHAR(30) NOT NULL,
	`Tipo` CHAR(1) NOT NULL,
	`Nome` VARCHAR(30) NOT NULL,
	`Descrizione` TEXT(5000),
	PRIMARY KEY (`Id`),
	FOREIGN KEY (`Email_A`) REFERENCES AMMINISTRATORE(`Email_U`)
) ENGINE = InnoDB;
\end{lstlisting}

\subsection{Partita}
Struttura della tabella \emph{PARTITA}:

\begin{lstlisting}
CREATE TABLE IF NOT EXISTS `PARTITA` (
	`Id` INT UNSIGNED NOT NULL,
	`Tipo_torneo` CHAR(1) NOT NULL,
	`Luogo` VARCHAR(30) NOT NULL,
	`Data` DATE NOT NULL,
	`Ora` TIME NOT NULL,
	PRIMARY KEY (`Id`)
) ENGINE = InnoDB;
\end{lstlisting}

\subsection{Referto}
Struttura della tabella \emph{REFERTO}:

\begin{lstlisting}
CREATE TABLE IF NOT EXISTS `REFERTO` (
	`Id` INT UNSIGNED NOT NULL,
	`Id_P` INT UNSIGNED NOT NULL,
	`Email_AR` VARCHAR(30) NOT NULL,
	`Ora_inizio` TIME NOT NULL,
	`Ora_fine` TIME NOT NULL,
	`Risultato` VARCHAR(10) NOT NULL,
	`Compilato` BOOLEAN NOT NULL,
	PRIMARY KEY (`Id`),
	FOREIGN KEY (`Id_P`) REFERENCES PARTITA(`Id`),
	FOREIGN KEY (`Email_AR`) REFERENCES ARBITRO(`Email_R`)
) ENGINE = InnoDB;
\end{lstlisting}

\subsection{Squadra}
Struttura della tabella \emph{SQUADRA}:

\begin{lstlisting}
CREATE TABLE IF NOT EXISTS `SQUADRA` (
	`Id` INT UNSIGNED NOT NULL,
	`Email_G` VARCHAR(30) NOT NULL,
	`Compilata` BOOLEAN NOT NULL,
	`Nome` VARCHAR(30) NOT NULL,
	`Descrizione` TEXT(5000),
	`Sede` VARCHAR(30),
	`Logo` VARCHAR(50),
	`Immagine` VARCHAR(50),
	`Nome_sponsor` VARCHAR(30),
	`Logo_sponsor` VARCHAR(50),
	PRIMARY KEY (`Id`),
	FOREIGN KEY (`Email_G`) REFERENCES GESTORE_SQUADRA(`Email_R`)
) ENGINE = InnoDB;
\end{lstlisting}

\subsection{Classifica}
Struttura della tabella \emph{CLASSIFICA}:

\begin{lstlisting}
CREATE TABLE IF NOT EXISTS `CLASSIFICA` (
	`Id_T` INT UNSIGNED NOT NULL,
	`Id_S` INT UNSIGNED NOT NULL,
	`Punti` INT UNSIGNED DEFAULT 0,
	`Vittorie` INT UNSIGNED DEFAULT 0,
	`Sconfitte` INT UNSIGNED DEFAULT 0,
	`Pareggi` INT UNSIGNED DEFAULT 0,
	`Partite` INT UNSIGNED DEFAULT 0,
	`Goal_fatti` INT UNSIGNED DEFAULT 0,
	`Goal_subiti` INT UNSIGNED DEFAULT 0,
	PRIMARY KEY (`Id_T`, `Id_S`),
	FOREIGN KEY (`Id_T`) REFERENCES TORNEO(`Id`),
	FOREIGN KEY (`Id_S`) REFERENCES SQUADRA(`Id`)
) ENGINE = InnoDB;
\end{lstlisting}

\subsection{Giocatore}
Struttura della tabella \emph{GIOCATORE}:

\begin{lstlisting}
CREATE TABLE IF NOT EXISTS `GIOCATORE` (
	`Id` INT UNSIGNED NOT NULL,
	`Id_S` INT UNSIGNED NOT NULL,
	`Attivo` BOOLEAN DEFAULT TRUE,
	`Numero_maglia` INT UNSIGNED,
	`Ruolo` VARCHAR(50) NOT NULL,
	`Nome` VARCHAR(30) NOT NULL,
	`Cognome` VARCHAR(30) NOT NULL,
	`Data_di_nascita` DATE NOT NULL,
	`Nazionalita` VARCHAR(20) NOT NULL,
	`Foto` VARCHAR(50) NOT NULL,
	`Descrizione` TEXT(5000),
	`Goal` INT UNSIGNED DEFAULT 0,
	`Ammonizioni` INT UNSIGNED DEFAULT 0,
	`Espulsioni` INT UNSIGNED DEFAULT 0,
	PRIMARY KEY (`Id`),
	FOREIGN KEY (`Id_S`) REFERENCES SQUADRA(`Id`)
) ENGINE = InnoDB;
\end{lstlisting}

\newpage

\subsection{Marcatore}
Struttura della tabella \emph{MARCATORE}:

\begin{lstlisting}
CREATE TABLE IF NOT EXISTS `MARCATORE` (
	`Id_R` INT UNSIGNED NOT NULL,
	`Id_G` INT UNSIGNED NOT NULL,
	`Goal` INT UNSIGNED DEFAULT 0,
	PRIMARY KEY (`Id_R`, `Id_G`),
	FOREIGN KEY (`Id_R`) REFERENCES REFERTO(`Id`),
	FOREIGN KEY (`Id_G`) REFERENCES GIOCATORE(`Id`)
) ENGINE = InnoDB;
\end{lstlisting}

\subsection{Partecipano}
Struttura della tabella \emph{PARTECIPANO}:

\begin{lstlisting}
CREATE TABLE IF NOT EXISTS `PARTECIPANO` (
	`Id_S` INT UNSIGNED NOT NULL,
	`Id_T` INT UNSIGNED NOT NULL,
	PRIMARY KEY (`Id_S`, `Id_T`),
	FOREIGN KEY (`Id_S`) REFERENCES SQUADRA(`Id`),
	FOREIGN KEY (`Id_T`) REFERENCES TORNEO(`Id`)
) ENGINE = InnoDB;
\end{lstlisting}

\subsection{Cartellino}
Struttura della tabella \emph{CARTELLINO}:

\begin{lstlisting}
CREATE TABLE IF NOT EXISTS `CARTELLINO` (
	`Id_R` INT UNSIGNED NOT NULL,
	`Id_G` INT UNSIGNED NOT NULL,
	`Numero` INT UNSIGNED DEFAULT 0,
	`Ammonizione` BOOLEAN,
	`Espulsione` BOOLEAN,
	PRIMARY KEY (`Id_R`, `Id_G`),
	FOREIGN KEY (`Id_R`) REFERENCES REFERTO(`Id`),
	FOREIGN KEY (`Id_G`) REFERENCES GIOCATORE(`Id`)
) ENGINE = InnoDB;
\end{lstlisting}

\newpage

\subsection{Scelto}
Struttura della tabella \emph{SCELTO}:

\begin{lstlisting}
CREATE TABLE IF NOT EXISTS `SCELTO` (
	`Id_P` INT UNSIGNED NOT NULL,
	`Email_AR` VARCHAR(30) NOT NULL,
	`Email_A` VARCHAR(30) NOT NULL,
	PRIMARY KEY (`Id_P`, `Email_AR`, `Email_A`),
	FOREIGN KEY (`Id_P`) REFERENCES PARTITA(`Id`),
	FOREIGN KEY (`Email_AR`) REFERENCES ARBITRO(`Email_R`),
	FOREIGN KEY (`Email_A`) REFERENCES AMMINISTRATORE(`Email_U`)
) ENGINE = InnoDB;
\end{lstlisting}

\subsection{Formazione}
Struttura della tabella \emph{FORMAZIONE}:

\begin{lstlisting}
CREATE TABLE IF NOT EXISTS `FORMAZIONE` (
	`Id_R` INT UNSIGNED NOT NULL,
	`Id_S` INT UNSIGNED NOT NULL,
	`Id_G` INT UNSIGNED NOT NULL,
	`Riserva` BOOLEAN NOT NULL,
	PRIMARY KEY (`Id_R`, `Id_S`, `Id_G`),
	FOREIGN KEY (`Id_R`) REFERENCES REFERTO(`Id`),
	FOREIGN KEY (`Id_S`) REFERENCES SQUADRA(`Id`),
	FOREIGN KEY (`Id_G`) REFERENCES GIOCATORE(`Id`)
) ENGINE = InnoDB;
\end{lstlisting}

\subsection{Giocano}
Struttura della tabella \emph{GIOCANO}:

\begin{lstlisting}
CREATE TABLE IF NOT EXISTS `GIOCANO` (
	`Id_P` INT UNSIGNED NOT NULL,
	`Id_S1` INT UNSIGNED NOT NULL,
	`Id_S2` INT UNSIGNED NOT NULL,
	`Id_T` INT UNSIGNED NOT NULL,
	`Nome_partita` VARCHAR(30) NOT NULL,
	PRIMARY KEY (`Id_P`, `Id_S1`, `Id_S2`, `Id_T`),
	FOREIGN KEY (`Id_P`) REFERENCES PARTITA(`Id`),
	FOREIGN KEY (`Id_S1`) REFERENCES SQUADRA(`Id`),
	FOREIGN KEY (`Id_S2`) REFERENCES SQUADRA(`Id`),
	FOREIGN KEY (`Id_T`) REFERENCES TORNEO(`Id`)
) ENGINE = InnoDB;
\end{lstlisting}

\section{Interrogazioni}
Tra le molte interrogazioni utilizzate dall'applicazione web riguardo il database precedente si riportano quelle che meritano un commento.

\subsection{INSERT}

\subsubsection*{Inserimento arbitro}

\begin{lstlisting}
UPDATE arbitro
SET 
Foto='Conversion.getDatabaseString(foto)', data_n='Conversion.getDatabaseString(data_n)',
Nazionalita='Conversion.getDatabaseString(nazionalita)', Carriera='Conversion.getDatabaseString(carriera)'
WHERE SSN=(SELECT SSN FROM utente WHERE email='"+Conversion.getDatabaseString(email)')
\end{lstlisting}

\subsection*{Eliminazione di un utente dall'applicazione}
L'amministratore ha la possibilità di togliere i privilegi di accesso all'applicazione ad un utente registrato. In realtà non viene eliminato dal database ma viene solo cambiata la sua visibilità attraverso l'attributo \emph{Attivo}.

\begin{lstlisting}
UPDATE utente
SET flag='N' 
WHERE  email='Conversion.getDatabaseString(email)'
\end{lstlisting}


\subsection*{Inserimento dei marcatori di una partita}
Nella creazione del referto viene utilizzata questa query per inserire in una tabella tutti i giocatori che hanno segnato dei goal nella partita a cui fa riferimento il referto. Una cosa simile è stata fatta per la cartella Cartellini.

\begin{lstlisting}
INSERT INTO marcatori (ID_Referto, SSN_Giocatore, Goal) 
VALUES ('Conversion.getDatabaseString(""+IDReferto)', 'Conversion.getDatabaseString(""+SSNGiocatore)', 'Conversion.getDatabaseString(""+goal)')
\end{lstlisting}

\subsection{UPDATE}

\subsection*{Primo accesso all'applicazione del gestore di una squadra}
Primo accesso all'applicazione del gestore di una squadra

\begin{lstlisting}
UPDATE squadra
SET    Nome_squadra = 'Conversion.getDatabaseString(nomeSquadra)',
Logo_squadra = Conversion.getDatabaseString(logoSquadra),
Immagine_squadra = 'Conversion.getDatabaseString(immagine Squadra)',
Nome_sponsor = 'Conversion.getDatabaseString(nomeSponsor)',
Logo_sponsor = 'Conversion.getDatabaseString(logoSponsor)',
Sede = 'Conversion.getDatabaseString(sede)',
Descrizione = 'Conversion.getDatabaseString(descrizione)',
flag = 'Y'
WHERE  SSN_Gestore = 'Conversion.getDatabaseString(""+gestore)'
\end{lstlisting}

\subsection*{Inserimento di un nuovo arbitro}
L'amministratore ha la possibilità di inserire un nuovo arbitro, a esso gli viene attribuita la lettera ``R'' (dall'inglese Refree) nell'attributo \emph{Tipo}. Le query sono tre perchè l'arbitro è un registrato che a sua volta è un cliente.

\subsubsection*{Inserimento utente}

\begin{lstlisting}
UPDATE utente
SET 
flag='Y', type='R', Nome='Conversion.getDatabaseString(nome)', Cognome='Conversion.getDatabaseString(cognome)',
Password='Conversion.getDatabaseString(password)'
WHERE email='Conversion.getDatabaseString(email)'
\end{lstlisting}

\subsubsection*{Inserimento registrato}

\begin{lstlisting}
UPDATE registrato
SET 
Telefono='Conversion.getDatabaseString(telefono)', Indirizzo='Conversion.getDatabaseString(indirizzo)',
SSN_Admin='Conversion.getDatabaseString(""+Admin)'
WHERE SSN=(SELECT SSN              FROM utente             WHERE email='Conversion.getDatabaseString(email)')
\end{lstlisting}

\subsection{SELECT}

\subsection*{Ricerca partite senza nessun arbitro}

\begin{lstlisting}
SELECT p.Data_partita, R.ID_Referto,g., r.flagRef, t.Nome AS nomeTorneo
FROM referto AS r, giocano AS g, torneo AS t, partita AS p
WHERE 	r.flagRef='N' AND
G.ID_Partita=r.ID_Partita AND
g.ID_Torneo=t.ID_Torneo AND
r.ID_Partita=p.ID_Partita AND
g.ID_Partita=p.ID_Partita AND
(G.ID_SquadraA IN
(SELECT ID_SQUADRA
FROM SQUADRA 
WHERE ssn_gESTORE='"+Conversion.getDatabaseString(""+gestore)  AND
ID_Squadra<>'0')
OR G.ID_SquadraB IN
(SELECT ID_SQUADRA
FROM SQUADRA
WHERE ssn_gESTORE='"+Conversion.getDatabaseString(""+gestore) AND
ID_Squadra<>'0')) 
AND R.ID_Referto NOT IN
(SELECT ID_Referto
FROM formazione 
WHERE ID_Squadra=(SELECT ID_Squadra
FROM squadra
WHERE SSN_Gestore='"+Conversion.getDatabaseString(""+gestore)))
ORDER BY ID_Referto
\end{lstlisting}

\subsection*{Ricerca dei referti assegnati ad un arbitro da un amministratore}
Quando l'amministratore del sistema deve sorteggiare un arbitro per una determinata partita riguardante una certa fase di un determinato torneo, deve poter visualizzare soltanto gli incontri per cui il sorteggio non sia già stato effettuato.

\begin{lstlisting}
SELECT 	Ref.ID_Referto, Ref.Risultato, Ref.Ora_inizio,
Ref.Ora_fine, Ref.ID_Partita, Ref.Risultato,
Ref.flagRef,U.Nome AS nomeArbitro,
U.cognome AS cognomeArbitro,T.ID_Torneo,
T.tipologia as tipoTorneo, T.Nome as nomeTorneo,
P.Data_Partita, P.Luogo
FROM 	referto AS Ref, utente AS U, registrato AS Reg,
giocano as G, torneo as T, partita as P
WHERE	U.Flag='Y' AND U.type='R' AND
U.SSN=Reg.SSN AND U.SSN=Ref.SSN_Arbitro
AND g.id_partita=Ref.ID_Partita
AND t.ID_Torneo=G.ID_Torneo
AND p.ID_Partita=G.ID_Partita
AND Reg.SSN_Admin='"+Conversion.getDatabaseString(""+Admin)'
ORDER BY Ref.ID_Referto 
\end{lstlisting}

\subsection*{Estrazione dei cartellini dei giocatori in una determinata partita}
Questa query viene utilizzata dall'utente pubblico quando va a visionare il referto relativo ad una determinata partita. Vengono estratte dalla tabella i cartellini di tutti i giocatori e le espulsioni e le ammonizioni relative ad una determinata partita. GLi \emph{Id} dei giocatori verranno poi confrontati con gli \emph{Id} dei giocatori delle due formazioni e se risultano uguali allora estraggo il flag ammonito/espulso e li visualizzo a schermo. Una query simile è stata fatta per la cartella Marcatori.

\begin{lstlisting}
SELECT g.Nome, g.Cognome, g.Foto, g.Foto, g.SSN_Gct, G.ID_Squadra, c.Flag_Ammonito as FlagAmmonito, c.Flag_Espulso as FlagEspulso
FROM giocatore AS g, cartellini_gialli AS C
WHERE C.SSN_Giocatore=G.SSN_Gct  AND C.ID_Referto='Conversion.getDatabaseString(""+IDReferto)'
AND (G.ID_Squadra='Conversion.getDatabaseString(""+IDSquadraA)' OR G.ID_Squadra='Conversion.getDatabaseString(""+IDSquadraB)')
\end{lstlisting}